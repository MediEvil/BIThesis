%%
% The SHUThesis Template for Bachelor Graduation Thesis
%
% 上海大学毕业设计(论文)—— 使用 XeLaTeX 编译
% 
% forked from spencerwooo/BIThesis original author is @SpencerWoo
% Copyright 2020 MediEvil
%%

% 章节支持、单面打印:ctexbook
\documentclass[UTF8,AutoFakeBold,AutoFakeSlant,zihao=-4,oneside,openany]{ctexbook}
\usepackage[a4paper,left=3cm,right=2.6cm,top=3.5cm,bottom=2.9cm]{geometry}
% 目前 29mm 最接近 Word 排版
\usepackage{xeCJK}
\usepackage{titletoc}
\usepackage{fontspec}
\usepackage{setspace}
\usepackage{graphicx}
\usepackage{fancyhdr}
\usepackage{pdfpages}
\usepackage{setspace}
\usepackage{booktabs}
\usepackage{multirow}
\usepackage{caption}
\usepackage{tikz}
\usepackage{etoolbox}
\usepackage{hyperref}
\usepackage{xcolor}
\usepackage{caption}
\usepackage{array}
\usepackage{amsmath}
\usepackage{amssymb}
\usepackage{pdfpages}

% 中文胡言乱语
\usepackage{zhlipsum}

% 英文胡言乱语
\usepackage{lipsum}

% 设置参考文献编译后端为 biber,引用格式为 GB/T7714-2015 格式
% 参考文献使用宏包见 https://github.com/hushidong/biblatex-gb7714-2015
\usepackage[
  backend=biber,
  style=gb7714-2015,
  gbalign=gb7714-2015,
  gbnamefmt=lowercase,
  doi=false,
  url=false
]{biblatex}

% 参考文献引用文件位于 misc/ref.bib
\addbibresource{misc/ref.bib}

% 西文字体默认为 Times New Roman
\setromanfont{Times New Roman}
% 论文题目字体为华文细黑
\setCJKfamilyfont{xihei}{STXihei}
\newcommand{\xihei}{\CJKfamily{xihei}}

% 在这里填写你的论文中英文题目
\newcommand{\thesisTitle}{上海大学本科生毕业设计(论文)题目}
\newcommand{\thesisTitleEN}{The Subject}

% 在这里填写你的相关信息
\newcommand{\deptName}{经济学院}
\newcommand{\majorName}{金融系}
\newcommand{\yourName}{吴主席}
\newcommand{\yourStudentID}{16120000}
\newcommand{\mentorName}{倪中新}

% 主题页面格式:BIThesis
\fancypagestyle{BIThesis}{
  % 页眉高度
  \setlength{\headheight}{20pt}
  % 页码高度(不完美,比规定稍微靠下 2mm)
  \setlength{\footskip}{14pt}

  \fancyhf{}
  % 定义页眉、页码
  \fancyhead[C]{\zihao{4}\ziju{0.08}\songti{上海大学本科生毕业设计(论文)}}
  \fancyfoot[C]{\songti\zihao{5} \thepage}
  % 页眉分割线稍微粗一些
  \renewcommand{\headrulewidth}{0.6pt}
}

% 设置章节格式
% 一级标题:黑体,小二号,加粗;间距:段前 0.5 行,段后 1 行;
\ctexset{chapter={
    name = {第,章},
    number = {\arabic{chapter}},
    format = {\heiti \bfseries \centering \zihao{-2}},
    aftername = \hspace{9bp},
    pagestyle = BIThesis,
    beforeskip = 8bp,
    afterskip = 32bp,
    fixskip = true,
  }
}

% 二级标题:黑体,小三号,加粗;间距:段前 0.5 行,段后 0 行;
\ctexset{section={
    number = {\thechapter.\hspace{4bp}\arabic{section}},
    format = {\heiti \raggedright \bfseries \zihao{-4}},
    aftername = \hspace{8bp},
    beforeskip = 20bp plus 1ex minus .2ex,
    afterskip = 18bp plus .2ex,
    fixskip = true,
  }
}

% 三级标题:黑体、四号、加粗;间距:段前 0.5 行,段后 0 行;
\ctexset{subsection={
    number = {\thechapter.\hspace{3bp}\arabic{section}.\hspace{3bp}\arabic{subsection}},
    format = {\heiti \bfseries \raggedright \zihao{4}},
    aftername = \hspace{7bp},
    beforeskip = 17bp plus 1ex minus .2ex,
    afterskip = 14bp plus .2ex,
    fixskip = true,
  }
}

% 设置目录样式
% 添加 PDF 链接
\addtocontents{toc}{\protect\hypersetup{hidelinks}}

% 解决「目录」二字的格式问题
\renewcommand{\contentsname}{
  \fontsize{16pt}{\baselineskip}
  \normalfont\heiti{目~~~~录}
  \vspace{-8pt}
}
% 定义目录样式
\titlecontents{chapter}[0pt]{\songti \zihao{-4}}
{\thecontentslabel\hspace{\ccwd}}{}
{\hspace{.5em}\titlerule*{.}\contentspage}
\titlecontents{section}[2\ccwd]{\songti \zihao{-4}}
{\thecontentslabel\hspace{\ccwd}}{}
{\hspace{.5em}\titlerule*{.}\contentspage}
\titlecontents{subsection}[4\ccwd]{\songti \zihao{-4}}
{\thecontentslabel\hspace{\ccwd}}{}
{\hspace{.5em}\titlerule*{.}\contentspage}

% 前置页面(原创性声明、中英文摘要、目录等)
\renewcommand{\frontmatter}{
  \pagenumbering{Roman}
  \pagestyle{BIThesis}
}

% 正文页面
\renewcommand{\mainmatter}{
  \pagenumbering{arabic}
  \pagestyle{BIThesis}
}

% 设置 caption 与 figure 之间的距离
\setlength{\abovecaptionskip}{11pt}
\setlength{\belowcaptionskip}{9pt}

% 设置图片的 caption 格式
\renewcommand{\thefigure}{\thechapter-\arabic{figure}}
\captionsetup[figure]{font=small,labelsep=space}

% 设置表格的 caption 与 table 之间的垂直距离
\captionsetup[table]{skip=2pt}

% 设置表格的 caption 格式
\renewcommand{\thetable}{\thechapter-\arabic{table}}
\captionsetup[table]{font=small,labelsep=space}

% 设置数学公式编号格式
\renewcommand{\theequation}{\arabic{chapter}-\arabic{equation}}

% 文档开始
\begin{document}

% 标题页面:如无特殊需要,本部分无需改动
%%
% The SHUThesis Template for Bachelor Graduation Thesis
%
% 上海大学毕业设计(论文)封面页 —— 使用 XeLaTeX 编译
%
% Copyright 2020 MediEvil
%%
% 封面
%
% 如无特殊需要,本页面无需更改

% Underline new command for student information
% Usage: \dunderline[<offset>]{<line_thickness>}
\newcommand\dunderline[3][-1pt]{{%
  \setbox0=\hbox{#3}
  \ooalign{\copy0\cr\rule[\dimexpr#1-#2\relax]{\wd0}{#2}}}}

% Cover Page
\begin{titlepage}
  \vspace*{19mm}
  \includegraphics[width=3cm]{images/bit_logo.png}
  
  \begin{center}
    \includegraphics[width=6cm]{images/header.png}

    \vspace*{-3mm}

    \zihao{-0}\textbf{\ziju{0.12}\songti{毕业设计(论文)}}
  \end{center}

  \vspace{16mm}

  \noindent\zihao{2}\textbf{题目:\xihei\thesisTitle}

  \vspace{3mm}

  % \begin{spacing}{1.2}
  %   \zihao{3}\selectfont{\textbf{\thesisTitleEN}}
  % \end{spacing}

  % \vspace{15mm}

  \flushleft
  \begin{spacing}{1.8}
    \hspace{27mm}\songti\zihao{3}\selectfont{学\hspace{11mm}院:\dunderline[-10pt]{1pt}{\makebox[78mm][c]{\deptName}}}

    \hspace{27mm}\songti\zihao{3}\selectfont{专\hspace{11mm}业:\dunderline[-10pt]{1pt}{\makebox[78mm][c]{\majorName}}}

    \hspace{27mm}\songti\zihao{3}\selectfont{学生姓名:\dunderline[-10pt]{1pt}{\makebox[78mm][c]{\yourName}}}

    \hspace{27mm}\songti\zihao{3}\selectfont{学\hspace{11mm}号:\dunderline[-10pt]{1pt}{\makebox[78mm][c]{\yourStudentID}}}

    \hspace{27mm}\songti\zihao{3}\selectfont{指导教师:\dunderline[-10pt]{1pt}{\makebox[78mm][c]{\mentorName}}}
  \end{spacing}

  \vspace{25mm}

  \centering
  \zihao{3}\ziju{0.5}\songti{\today}
\end{titlepage}


% 前置页面定义
\frontmatter
% 原创性声明:如无特殊需要,本部分无需改动
% 更改为 PDF 页面插入,如需要添加内容,可考虑先用 Word 制作再覆盖 misc/1_originality.pdf
% \includepdf{misc/1_originality.pdf}
\newpage
%\input{misc/1_originality.tex}
% 摘要:在摘要相应的 TeX 文件处进行摘要部分的撰写
%%
% The SHUThesis Template for Bachelor Graduation Thesis
%
% 上海大学毕业设计(论文)中英文摘要 —— 使用 XeLaTeX 编译
%
% Copyright 2020 MediEvil
%%

% 中英文摘要章节
\topskip=0pt
\zihao{-4}

\vspace*{-7mm}

\begin{center}
  \heiti\zihao{-2}\textbf{\thesisTitle}
\end{center}

\vspace*{2mm}

\addcontentsline{toc}{chapter}{摘~~~~要}
{\let\clearpage\relax \chapter*{\textmd{摘~~~~要}}}
\setcounter{page}{1}

\vspace*{1mm}

\setstretch{1.53}
\setlength{\parskip}{0em}

% 中文摘要正文从这里开始
\zhlipsum[1-2]

\vspace{4ex}\noindent\textbf{\heiti 关键词:上海大学;本科生;毕业设计(论文)}
\newpage

% 英文摘要章节
\topskip=0pt

\vspace*{2mm}

\begin{spacing}{0.95}
  \centering
  \heiti\zihao{3}\textbf{\thesisTitleEN}
\end{spacing}

\vspace*{17mm}

\addcontentsline{toc}{chapter}{Abstract}
{\let\clearpage\relax \chapter*{
  \zihao{-3}\textmd{Abstract}\vskip -3bp}}
\setcounter{page}{2}

\setstretch{1.53}
\setlength{\parskip}{0em}

% 英文摘要正文从这里开始
\lipsum[1]

\vspace{3ex}\noindent\textbf{Key Words: SHU Undergraduate; Graduation Project (Thesis)}
\newpage

% 目录:如无特殊需要,本部分无需改动
%%
% The SHUThesis Template for Bachelor Graduation Thesis
%
% 上海大学毕业设计(论文)目录 —— 使用 XeLaTeX 编译
%
% Copyright 2020 MediEvil
%%
% 如无特殊需要,本页面无需更改

% 目录开始

% 调整目录行间距
\renewcommand{\baselinestretch}{1.35}
% 目录
\tableofcontents
\newpage


% 正文开始
\mainmatter
% 正文 22 磅的行距
\setlength{\parskip}{0em}
\renewcommand{\baselinestretch}{1.53}

% 第一章
%%
% The SHUThesis Template for Bachelor Graduation Thesis
%
% 上海大学毕业设计(论文)第一章节
%
% Copyright 2020 MediEvil
%%
%
% 第一章节

\chapter{一级题目}

\section{二级题目}

\subsection{三级题目}
% 这里插入一个参考文献,仅作参考
\zhlipsum

这是一个参考文献\cite{yuFeiJiZongTiDuoXueKeSheJiYouHuaDeXianZhuangYuFaZhanFangXiang2008}

这里放了一张图\label{标题序号}
\begin{figure}[htbp]
  \vspace{13pt} % 调整图片与上文的垂直距离
  \centering
  \includegraphics[]{images/bit_logo.png}
  \caption{标题序号}\label{标题序号} % label 用来在文中索引
\end{figure}

这里放了一张表\label{统计表}
\begin{table}[htbp]
  \linespread{1.5}
  \zihao{5}
  \centering
  \caption{统计表}\label{统计表}
  \begin{tabular}{*{5}{>{\centering\arraybackslash}p{2cm}}}
    \hline
    项目    & 产量    & 销量    & 产值   & 比重    \\ \hline
    手机    & 1000  & 10000 & 500  & 50\%  \\
    计算机   & 5500  & 5000  & 220  & 22\%  \\
    笔记本电脑 & 1100  & 1000  & 280  & 28\%  \\ \hline
    合计    & 17600 & 16000 & 1000 & 100\% \\ \hline
  \end{tabular}
\end{table}

这里插入一个公式
\begin{equation}
    LRI=1\ ∕\ \sqrt{1+{\left(\frac{{\mu }_{R}}{{\mu }_{s}}\right)}^{2}{\left(\frac{{\delta }_{R}}{{\delta }_{s}}\right)}^{2}}
\end{equation}

% 在这里添加第二章、第三章……TeX 文件的引用
% \input{chapters/2_chapter2.tex}
% \input{chapters/3_chapter3.tex}

% 结论:在结论相应的 TeX 文件处进行结论部分的撰写
%%
% The SHUThesis Template for Bachelor Graduation Thesis
%
% 上海大学毕业设计(论文)结论 —— 使用 XeLaTeX 编译
%
% Copyright 2020 MediEvil
%%

\addcontentsline{toc}{chapter}{结~~~~论}
\chapter*{\vskip 10bp\textmd{结~~~~论} \vskip -6bp}

\zhlipsum[4]


% 参考文献:如无特殊需要,参考文献相应的 TeX 文件无需改动,添加参考文献请使用 BibTeX 的格式
%   添加至 misc/ref.bib 中,并在正文的相应位置使用 \cite{xxx} 的格式引用参考文献
\input{misc/4_reference.tex}
% 附录:在附录相应的 TeX 文件处进行附录部分的撰写
\input{misc/5_appendix.tex}
% 致谢:在致谢相应的 TeX 文件处进行致谢部分的撰写
%%
% The SHUThesis Template for Bachelor Graduation Thesis
%
% 上海大学毕业设计(论文)原致谢 —— 使用 XeLaTeX 编译
%
% Copyright 2020 MediEvil
%%

\addcontentsline{toc}{chapter}{致~~~~谢}
\chapter*{\vskip 10bp \textmd{致~~~~谢} \vskip -6bp}

值此论文完成之际,首先向我的导师……

\textcolor{blue}{致谢正文样式与文章正文相同:宋体、小四;行距:22 磅;间距段前段后均为 0 行。阅后删除此段。}


\end{document}
